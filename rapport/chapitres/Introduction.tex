\chapter*{General Introduction}
\addcontentsline{toc}{chapter}{General Introduction}
\raggedright
For an institution to efficiently manage its administrative processes, it must rely on optimized systems that ensure fluidity, transparency, and reliability. In the current context, marked by an increasing digital transformation, the automation of administrative tasks has become an essential lever to improve service management, reduce processing times, and minimize human errors.

\vspace{0.3cm}
The efficiency of a process depends not only on the technology deployed, but also on its alignment with the specific needs of the institution and its users. Thus, for an organization such as the National Social Security Fund \textbf{(CNSS)}, which processes a large number of files daily, it is crucial to have a customized digital solution to manage the submission and tracking of files.

\vspace{0.3cm}
Digitalization of processes, in addition to improving operational efficiency, simplifies interactions between different stakeholders, ensures better traceability, and guarantees secure access to information. By reducing repetitive and time-consuming tasks, this transformation allows the institution to focus more on its core mission and provide high-quality services.

\vspace{0.3cm}
However, although several generic tools are available for file submission management, CNSS requires a customized solution aligned with its organizational and regulatory specificities. This is the context in which our final year project, titled \textbf{"Design and Digitalization of the Online File Submission Process"} takes place.

\vspace{0.2cm}
This report is organized into six chapters:
\begin{itemize}
    \item[\textbullet] \textbf{Chapter 1: Project Study}\\
    Presents the host organization, project context and objectives, state-of-the-art analysis, critique of the existing system, proposed solution, and adopted methodology.
    \vspace{0.5cm}
    
    \item[\textbullet] \textbf{Chapter 2: Planning and Architecture}\\
    Details functional and non-functional requirements, overall use case diagram, and task distribution within the project team.
    \vspace{0.5cm}
    
    \item[\textbullet] \textbf{Chapter 3: Sprint 1 – User and Submitted File Management}\\
    Covers the first development phase focused on user identification and file submission management.
    \vspace{0.5cm}
    
    \item[\textbullet] \textbf{Chapter 4: Sprint 2 – Tracking, Notifications and Statistics}\\
    Details the second development phase dedicated to managing validation processes, tracking, and stakeholder communication.
    \vspace{0.5cm}
    
    \item[\textbullet] \textbf{Chapter 5: Sprint 3 - User Access and Service Request Foundation}\\
    Describes the implementation of user authentication, password recovery, service request submission, email notifications, news management, and identity document validation.
    \vspace{0.5cm}
    
    \item[\textbullet] \textbf{Chapter 6: Closure Phase}\\
    Presents the tools and technologies used, results obtained, and feedback on the developed solution.
\end{itemize}
\chapter*{General Introduction}
\addcontentsline{toc}{chapter}{General Introduction} 

\Large For an institution to efficiently manage its administrative processes, it must rely on optimized systems that ensure fluidity, transparency, and reliability. In the current context, marked by an increasing digital transformation, the automation of administrative tasks has become an essential lever to improve service management, reduce processing times, and minimize human errors.

The efficiency of a process depends not only on the technology deployed, but also on its alignment with the specific needs of the institution and its users. Thus, for an organization such as the National Social Security Fund \textbf{(CNSS)}, which processes a large number of files daily, it is crucial to have a customized digital solution to manage the submission and tracking of files.

Digitalization of processes, in addition to improving operational efficiency, simplifies interactions between different stakeholders, ensures better traceability, and guarantees secure access to information. By reducing repetitive and time-consuming tasks, this transformation allows the institution to focus more on its core mission and provide high-quality services.

However, although several generic tools are available for file submission management, CNSS requires a customized solution aligned with its organizational and regulatory specificities. This is the context in which our final year project, titled \textbf{“Design and Digitalization of the Online File Submission Process"} takes place.

This report is structured into five chapters:

\begin{itemize}
    \item \textbf{Chapter 1: Project Study} – This chapter presents the host organization, the context and objectives of the project, a state-of-the-art analysis, a critique of the existing system, as well as the proposed solution and adopted methodology.
    \item \textbf{Chapter 2: Planning and Architecture}\\ - Details functional and non-functional requirements, the overall use case diagram, and the distribution of tasks within the project team.
    \item \textbf{Chapter 3: Sprint 1 – User and Submitted File Management} - 
  This chapter covers the first development phase, focusing on user identification and file submission management.
    \item \textbf{Chapter 4: Sprint 2 – Tracking, Notifications and Statistics} - Covers the second development phase dedicated to managing validation processes, tracking, and communication with the relevant parties.
    \item \textbf{Chapter 5: Sprint 3 - User Access and Service Request Foundation} - The system enables secure user authentication, password recovery, and service request submission. Users receive email updates and can view news, while central managers handle news management. Identity documents are validated through a back-end process
    \item \textbf{Chapter 6: Closure Phase}  –  This final chapter presents the tools and technologies used, the results obtained, and the feedback on the developed solution.
\end{itemize}